\documentclass{article}
\usepackage[margin=1.4in]{geometry}
\usepackage{graphicx}
\usepackage{physics}

\usepackage{placeins}

\let\Oldsection\section
\renewcommand{\section}{\FloatBarrier\Oldsection}

\let\Oldsubsection\subsection
\renewcommand{\subsection}{\FloatBarrier\Oldsubsection}

\let\Oldsubsubsection\subsubsection
\renewcommand{\subsubsection}{\FloatBarrier\Oldsubsubsection}

\title{Violation of CHSH inequality by polarization-entangled photon pairs}
\author{Luke Burns}
\date{13 May 2016}

\begin{document}

\maketitle

\begin{abstract}
  Linearly polarized photons produced by a 407~nm pump laser undergo Type-I Spontaneous Parametric Down Conversion (SPDC), facilitated by two Beta Barium Borate (BBO) crystals, yielding pairs of entangled 814~nm photons with correlated polarizations. These photons are passed through polarizers, and coincidences are counted for varied polarization configurations. A test of the CHSH Bell's inequality finds a violation by 8 standard deviations, with $S=2.56 \pm 0.07$.
\end{abstract}

\section{Introduction}

Quantum entanglement is a hallmark phenomenon of quantum mechanics. In 1935, Albert Einstein, Boris Podolsky, and Nathan Rosen (EPR) authored a paper entitled ``Can Quantum-Mechanical Description of Physical Reality Be Considered Complete?'', in which they argued that the wavefunction formulation of quantum mechanics either violates the basic physical principle of locality, which demands that physical objects only affect one another if they are interacting in some medium through which material propagates no faster than the speed of light, or is incomplete. EPR concluded that quantum mechanics must be incomplete and professed their belief in the possibility in a theory which is local and accounts for all the hidden variables not present in quantum mechanics.\cite{einstein}

In 1964, John Stewart Bell drew up a set of inequalities that so called local hidden variable theories must obey. The local hidden variable theories obeying these inequalities rested upon two principles: locality and realism, which was tacitly assumed by EPR in their argument. The principle of realism demands that properties of a system exist independently of their measurement. If these inequalities are demonstrably violated, then physicists would be forced to abandon at least one of these principles.\cite{bell}

In 1966, Bell proved the Bell-Kochen-Specker theorem, which elaborated upon the issue of realism, showing that there is a contradiction between thinking that hidden properties corresponding to observables have definite values at any given time and the values of those properties are independent of the device used to measure them.\cite{bell2}

In the early 1980s, Alain Aspect performed experiments using entangled photons and confirmed statistics which violated Bell's inequalities, hence definitively eliminating a local and real picture of nature.\cite{aspect} Since then, many inequalities have been proposed and loopholes tested, and experiments continue to validate quantum mechanics.

The Copenhagen interpretation opts to abandon both locality and realism, but there do exist formulations which opt to preserve locality and abandon realism, or vice versa, and remain consistent in all predictions. An example of the former kind of theory is Relational Quantum Mechanics (RQM), in which the properties of a system exist only in reference to another system, i.e. relationally.\cite{rovelli} An example of the latter kind of theory is de Broglie-Bohm theory, in which locality is abandoned, but the theory is completely deterministic and all properties are considered real.\cite{bohm} There are many formulations of quantum mechanics, just as there are many formulations of classical mechanics (e.g. Newtonian, Lagrangian, Hamiltonian, Liouville). Dan Styer has a good review of nine different formulations in \cite{styer}. It is important to note that, which principle one abandons in reasoning about quantum mechanics is entirely up to one's preference, to the same extent that working in the Lagrangian or Hamiltonian formulation of Classical mechanics is up to one's preference.

Here we present the details of an experiment in which photons are entangled via Spontaneous Parametric Downconversion (SPDC), passed through polarizers, and detected for coincidences for various polarizer configurations. A test of the CHSH Bell's inequality finds $S=2.56 \pm 0.07$, which indicates a violation by 8 standard deviations.

\section{Theory}

\subsection{Downconversion}

SPDC is a process in which a nonlinear material is used to covert one photon into two photons, each with half the energy. In Type-I SPDC, downconverted photons are polarized in the same direction as each other but perpendicular to the incident photon.

Due to dispersion, downconverted photon pairs propagate at a different speed than the incident photon, disrupting phase coherence of incident and downconverted photons, dramatically reducing the efficiency of the conversion process. Furthermore, SPDC is stimulated by vacuum fluctuations at random, leaving the conversion efficiency is on the order of 1 in $10^{12}$ incident photons.\cite{pors}

This problem can be resolved by use of birefringent materials, in which the refractive index is dependent on polarization. Using birefringence, one can maximize the efficiency of the conversion process by satisfying the phase matching condition

\begin{equation}
  \frac{\sin^2{\theta}}{n_e(\lambda)^2} + \frac{\cos^2{\theta}}{n_o(\lambda)^2} = \frac{\sec^2{\mu}}{n_o(2\lambda)^2}, \label{eq:phase}
\end{equation}

where $2 \mu$ is the angle between downconverted photons, $\theta$ is the angle the incident photon makes with the first crystal's optic axis, $n_e$ and $n_0$ are the wavelength($\lambda$)-dependent refractive indices for the BBO.

In this experiment, a pair of Beta Barium Borate (BBO) crystals, oriented orthogonally to one another, with cut $\theta=30^{\circ}$ facilitate Type-I SPDC of linearly polarized, 407~nm photons. With a BBO of this cut and photons of this energy, the angle at which the downconverted photons propagate through the crystals is $\mu = 3.2^{\circ}$. Each propagates at the angle $\mu$ relative to the direction of the incident photon, because the two downconverted photons have the same energy (a wavelength of 814~nm). 

Only the polarization component parallel to the optic axis of the BBO crystals contributes to SPDC. Incident photons are polarized at $\theta = 45^{\circ}$ using a half-wave plate so that downconverted photons are produced with polarizations $\theta = 0^{\circ}$ (horizontal) and $\theta = 90^{\circ}$ (vertical) with equal probability, half of them downconverting through the first BBO and half through the second.

\subsection{Entangled States}

A pair of downconverted photons are described by the state 

\begin{equation}
  \ket{\psi_{DC}} = \cos(\theta) \ket{H}_1 \ket{H}_2 + e^{i\phi} \sin(\theta) \ket{V}_1 \ket{V}_2,\label{eq:dc}
\end{equation}

where $\ket{A}_1 \ket{B}_2$ refers to a composite state of two spatially separated photons, one with polarization $A$ and one with polarization $B$, given an incident photon polarized at angle $\theta$ relative to the optic axis of the first crystal with a phase shift $\phi$ between vertical and horizontal polarization components. From now on indices are left off, because they are implicit in the ordering of the product.

With $\theta = 45^{\circ}$ and $\phi = 0$, the state is considered ``maximally entangled,'' and the pair of downconverted photons is described by:\cite{dehlinger}

\begin{equation}
  \ket{\psi_{\text{EPR}}} = \frac{1}{\sqrt{2}}(\ket{H} \ket{H} + \ket{V} \ket{V}). \label{eq:epr}
\end{equation}

In the basis of the polarizers at angle $\gamma$, the  polarization of the photons are

\begin{equation}
  \ket{H_\gamma} = \cos(\gamma) \ket{H} + \sin(\gamma) \ket{V} \label{eq:h}
\end{equation}

and 

\begin{equation}
  \ket{V_\gamma} = \cos(\gamma) \ket{V} - \sin(\gamma) \ket{H}. \label{eq:v}
\end{equation}

The probability that photon 1 is detected at a polarizer angle $\alpha$ and photon 2 at a polarizer angle $\beta$

\begin{equation}
  P_{VV}(\alpha, \beta) = |\bra{V_\alpha}\bra{V_\beta} \ket{\psi_{\text{EPR}}}|^2 = \frac{1}{2} \cos^2(\beta-\alpha), \label{eq:quantum}
\end{equation}

where $VV$ indicates the coincidence of photons both oriented vertically relative to the optic axis. There are 3 other probabilities $P_{HH}, P_{HV}, P_{VH}$, which refer to the coincidence probability of photons oriented horizontally relative to the optic axis, the coincidence probability of photon 1 being oriented horizontally and photon 2 being oriented vertically, and the coincidence probability of photon 1 being oriented vertically and photon 2 being oriented horizontally respectively. These are given by inner products analogous to \ref{eq:quantum}.

Note, that this is markedly different from one might expect from Malus's law. In particular, for $\alpha = 45^\circ$ and $\beta = -45^\circ$, Malus's law predicts a $50\%$ coincidence rate, whereas Equation \ref{eq:quantum} predicts a $0\%$ coincidence rate. Furthermore, the Equation \ref{eq:quantum} only depends on the difference in angle between the polarizers $\alpha - \beta$. This result places strict constraints on the kinds of theories that can be used to make the same predictions.

\subsection{Local Realist Theories}

Bell demonstrated that local realist (local hidden variable) theories must obey statistics that quantum mechanics does not. The CHSH inequality, named after its authors John Clauser, Michael Horne, Abner Shimony, and Richard Holt, is a reformulation that opened up tests to a wider range of experiments.\cite{clauser}

It states that the quantity 

\begin{equation}
  S = E(a,b) + E(a', b') + E(a', b) - E(a, b'),
\end{equation}

for four angles $a,a',b,b'$, must obey

\begin{equation}
  \abs{S} \leq 2.\label{eq:s}
\end{equation}

The correlation measure $E(\alpha, \beta)$ is theoretically given by 

\begin{equation}
  E(\alpha, \beta) = P_{VV}(\alpha, \beta) + P_{VV}(\alpha, \beta) - P_{HV}(\alpha, \beta) - P_{VH}(\alpha, \beta)
\end{equation}

and experimentally given by

\begin{equation}
  E(\alpha, \beta) = \frac{N(\alpha, \beta) + N(\alpha_\perp, \beta_\perp) - N(\alpha_\perp, \beta) - N(\alpha, \beta_\perp)}{N(\alpha, \beta) + N(\alpha_\perp, \beta_\perp) + N(\alpha_\perp, \beta) + N(\alpha, \beta_\perp)}, \label{eq:test}
\end{equation} 

where $N(\alpha, \beta)$ is the number of coincidences for polarizer configuration $(\alpha, \beta)$ and $\gamma_\perp = \gamma \pm 90^\circ$. In total, $S$ depends on coincidence measurements for 16 different polarizer configurations. For certain sets of configurations, quantum mechanics violates this inequality.\cite{dehlinger} In particular, $a = 45^\circ, a' = 0^\circ$ and $b = 22.5^\circ, b' = -22.5^\circ$ gives the maximum violation $S = 2 \sqrt{2}$.

\section{Experimental Procedure}

A HeNe pump laser is used to produce a beam of 407~nm photons which pass through a quarter wave plate (QWP), a half wave plate (HWP), and a pair of BBO crystals. The BBO crystals are used to facilitate Type-I SPDC, yielding pairs of entangled 814~nm photons with correlated polarizations for a small fraction of the incident light. The entangled pairs are passed through a pair of polarizers set to angles $(\alpha, \beta)$, as depicted in Figure \ref{fig:setup}, mounted at the end of two adjustable rails, before passing through narrow band filters and propagating through fiber optic cables into a single photon counting module (SPCM). The SPCM determines coincidences, with a coincidence window of $25$~ns, which are then recorded onto a computer.

\begin{figure}[!h]
  \centering
  \includegraphics[width=0.5\textwidth]{setup}
  \caption{A HeNe pump laser produces a beam of 407~nm photons which pass through a quarter wave plate (QWP), a half wave plate (HWP), and a pair of BBO crystals. Pairs of entangled 814~nm photons with correlated polarizations are produced via Spontaneous Parametric Downconversion (SPDC) for a small fraction of incident light. Entangled pairs pass through a pair of polarizers configured to angles $(\alpha, \beta)$, as depicted above, mounted at the end of two adjustable rails, before passing through narrow band filters and propagating through fiber optic cables into a single photon counting module (SPCM). The SPCM determines coincidences, which are recorded onto a computer. \label{fig:setup}}
\end{figure}

We began preparing this setup by equalizing the heights of the mirrors and BBO crystals. The BBO crystals were aligned normal to the laser, and the fiber optic collector surfaces were aligned normal to downconverted photons. Polarizers were added. We calibrated polarizers by minimizing the visible light through each. For the upper path (as shown in Figure \ref{fig:setup}), this was given by $\alpha = 171^\circ$, and for the lower path, $\beta = 157^\circ$, with a $14^\circ$ difference.

A HWP was placed before the BBO crystals, set to angle $8^\circ\pm0.5^\circ$ which was selected by equalizing the coincidences for $(0, 0)$ and $(90, 90)$. This ensured that incident photons were polarized at $\theta = 45^{\circ}$ relative to the BBO crystals' optic axes, so that Equation \ref{eq:dc} reduces to

\begin{equation}
  \ket{\psi_{DC}} = \frac{1}{\sqrt{2}}(\ket{H} \ket{H} + e^{i\phi} \ket{V} \ket{V}).\label{eq:dc1}
\end{equation}

A QWP was placed before the HWP, set to angle $143.5^\circ\pm0.5^\circ$, which was selected by equalizing the coincidences for $(\alpha, \beta) = (45, -45)$ and $(0, 90)$. This ensured that the phase difference between components of incident light, due to the pump not producing perfectly linearly polarized light, dispersion, and birefringence, was corrected to $\phi = 0^\circ$. Equation \ref{eq:dc1} then reduces to the maximally entangled state

\begin{equation}
  \ket{\psi_{EPR}} = \frac{1}{\sqrt{2}}(\ket{H} \ket{H} + \ket{V} \ket{V}).\label{eq:dc2}
\end{equation}

Coincidences were counted for 12 minute intervals and 16 different polarizer configurations selected in order to test the CHSH inequality.

\section{Results and Conclusion}

Table \ref{table} shows coincidences for 16 polarizer configurations which provide a test of the CHSH inequality. Using Equations \ref{eq:s} and \ref{eq:test}, and the choices $a = 45^\circ, a' = 0^\circ$ and $b = 22.5^\circ, b' = -22.5^\circ$, we find $S = 2.56 \pm 0.07$. The uncertainty is calculated using the standard error for Poisson statistics given by the square root of the count.

\begin{table}[!h]
\centering
\caption{Photon coincidence counts for 12 minute intervals and varied polarizer configurations $(\alpha, \beta)$. These measurements demonstrate a violation of the CHSH inequality, with $S=2.56 \pm 0.07$.}
\label{table}
\begin{tabular}{ccc}
$\alpha$ (deg) & $\beta$ (deg) & Coincidences \\ \hline
    -45  & -22.5 & 490          \\
    -45  & 22.5  & 107          \\
    -45  & 67.5  & 92           \\
    -45  & 112.5 & 532          \\
    0    & -22.5 & 531          \\
    0    & 22.5  & 540          \\
    0    & 67.5  & 109          \\
    0    & 112.5 & 133          \\
    45   & -22.5 & 134          \\
    45   & 22.5  & 554          \\
    45   & 67.5  & 561          \\
    45   & 112.5 & 208          \\
    90   & -22.5 & 73           \\
    90   & 22.5  & 94           \\
    90   & 67.5  & 566          \\
    90   & 112.5 & 519
\end{tabular}
\end{table}

$S$ violates Equation \ref{eq:s} by 8 standard deviations and definitively eliminates the possibility of a local and real theory which makes the same predictions as Equation \ref{eq:quantum}, reinforcing the results of hundreds of other entanglement experiments.

\pagebreak

\begin{thebibliography}{9}
  \bibitem{einstein}
    Einstein, A; Podolsky, B; Rosen, N (1935). 
    \emph{Can Quantum-Mechanical Description of Physical Reality be Considered Complete?}
    Physical Review 47 (10): 777.
  \bibitem{bell}
    Bell, J. S. (1964). 
    \emph{On the Einstein Podolsky Rosen Paradox}. 
    Physics 1 (3): 195.
  \bibitem{bell2}
    Bell, J.S. (1966). 
    \emph{On the problem of hidden variables in quantum mechanics}. 
    Reviews of Modern Physics (38): 447.
  \bibitem{aspect}
    Aspect, A; Dalibard, J; Roger, G (1981).
    \emph{Experimental Tests of Realistic Local Theories via Bell's Theorem}.
    Phys. Rev. Lett. (47) 460.
  \bibitem{rovelli}
    Smerlak, M; Rovelli, C (2007).
    \emph{Relational EPR.}
    Found. Phys. (37): 427.
  \bibitem{bohm}
    Bohm, David (1952). \emph{A Suggested Interpretation of the Quantum Theory in Terms of `Hidden Variables' I}. Physical Review 85: 166–179.
  \bibitem{styer}
    Styer, D; et al (2002).
    \emph{Nine formulations of quantum mechanics}.
    Am. J. Phys. 70: 288. 
  \bibitem{pors}
    Pors, J.B. (2011).
    \emph{Entangling light in high dimensions}. 
    Doctoral Thesis, Leiden University.
  \bibitem{dehlinger}
    Dehlinger, D; Mitchell, M.W. (2002).
    \emph{Entangled photons, nonlocality and Bell inequalities in the undergraduate laboratory}.
    Am. J. Phys. (70): 903.
  \bibitem{clauser}
    Clauser, et al (1969). 
    \emph{Proposed experiment to test local hidden-variable theories}.
    Phys. Rev. Lett. 23 (15): 880.

\end{thebibliography}

\end{document}